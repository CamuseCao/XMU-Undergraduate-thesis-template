% !TeX root = ../XMU.tex
\chapter{模板的使用说明}{The usage guide of this template}

\section{使用的前提}{Prediction}

为了使用该模板,需要安装一个TeX的发行版本。可以选择 \TeX{live} 或者 Mik\TeX{} ,他们都是跨平台的。而 \TeX{live} 打包了比较多的宏包,较为庞大,Mik\TeX{} 则是临时下载没有的宏包。这里我推荐使用 Mik\TeX{} 。但是对于 Mac,推荐使用 Mac\TeX{} ,它是为 Mac 定制的发行版本,应该比较合适。特别提醒 CTeX 套装无法运行该模板。关于编译方式需选择 Xe\LaTeX{},否则无法正常编译该模板。

参考文献使用 bibtex 的方式实现,根据 gbt7714-2015 的要求,有三种格式可以选择,分别是 numerical, numbers, authoryear 。大家根据学院的要求选择,本模板默认采用 super 的格式。可以通过  \verb|\documentclass[bibstyle=numbers]{Settings/XMUthesis}| 来进行选择。

字体的话考虑到不同系统的问题,不同系统采用了不同的自己配置,本模板基本采用 ctex 文档类提供的接口来配置字体。通过 \verb|\documentclass[font=empty]{Settings/XMUthesis}| 的方式调用。可以选择的包括 \verb|empty, adobe, fandol, founder, mac, macnew, macold, | \\ \verb|ubuntu,windows| 等。选择 \verb|adobe, fandol, founder| 的需要自行安装字体,不同选项所需的字体可以查看\verb|fonts/ReadMe.txt| 。 empty 选项可以自己根据系统判断自己的配置,该功能是 ctex 宏集实现的,本模板只是统一提供一个接口,其它几个可以根据自己的系统选择。详情可以通过 \verb|texdoc ctex| 第七页自行查看。如果对字体不满意的,可以选择 \verb|empty| 选项,然后自己配置字体。

由于学校要求英文使用 Times New Roman 和 Arial 字体,对于 Linux 用户,它们常常不是默认安装在系统中的字体,因此需要用户自行安装这两个字体。

\begin{table}[ht!]
  \centering
  \caption{已加载的宏包}
    \begin{tabular}{ccccc}\toprule
    algorithm     & etexcmds        & hopatch      & kvsetkeys   & refcount           \\\midrule
    algorithmicx  & etoolbox        & hycolor      & letltxmacro & rerunfilecheck     \\\midrule
    algpseudocode & filehook        & hyperref     & lstlang1    & stringenc          \\\midrule
    atbegshi      & float           & ifluatex     & lstmisc     & unicode-math       \\\midrule
    atveryend     & gbt7714         & ifthen       & ltxcmds     & unicode-math-xetex \\\midrule
    auxhook       & gettitlestring  & infwarerr    & nameref     & uniquecounter      \\\midrule
    bigintcalc    & hobsub          & intcalc      & natbib      & url                \\\midrule
    bitset        & hobsub-generic  & kvdefinekeys & pdfescape   & xcolor-patch       \\\midrule
    cleveref      & hobsub-hyperref & kvoptions    & pdftexcmds  & xeCJK-listings     \\\midrule
    ctex          &                 &              &             &                    \\\bottomrule
    \end{tabular}%
\end{table}%


\section{几点说明}{Some notes}

为了正确使用该模板,请按照提示安装好可使用的 \TeX{} 发行版本。因为论文内容比较多,因此采取了分文件的方式来构成该文档。主文档为 XMU.tex 。 Figure 文件夹是存放图片的文件夹,该文件夹已经加入图片文件夹的位置,插入图片是无需多加路径,直接用文件名即可。 Setting 文件夹是放置模板和宏包的文件夹,使用者最好不要更改里面的东西。而你需要编辑的仅有 Body 文件夹下的文件。

该模板是在厦门大学博士学位论文模板的基础上修改得到的,因为本科论文与博士学位论文的要求差别比较的,所以定制了该模板。由于本人水平有限,因此该模板写的并不好,但是应该勉强能够满足毕业论文的要求。但是仍然可能有许多错误的地方,希望各位使用者如果能发现错误之处能够提出。可以给我法邮件或者直接在 github 上面提 issue 。欢迎大家的参与,共同完善母校的模板。

由于本人是一名理科生,对文科的同学毕业论文的额外需求可能了解不多。虽说文科生用这个模板的可能性比较小,如果有人用,有额外的需求也可以提出。

联系方式:
邮箱: \href{mailto:camusecao@gmail.com}{camusecao@gmail.com}

github项目的地址 : \href{https://github.com/CamuseCao/XMU-Undergraduate-thesis-template}{XMU-Undergraduate-thesis-template}
